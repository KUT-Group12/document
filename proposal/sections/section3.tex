\section{機能概要・前提条件・制約事項}
\subsection{機能概要}
\subsubsection{一般向け}
\begin{itemize}[itemsep=10pt]
    \item 会員機能
    \begin{itemize}[itemsep=10pt]
        \item 新規会員登録 \mbox{}\\
        一般会員がGoogleアカウントを用いて会員になる機能.また,新規会員登録の際に利用規約を提示し,規約に従う場合にのみ新規登録を行う.
        \item ログイン・ログアウト \mbox{}\\
        ログイン機能では一般会員がGoogleアカウントを用いた認証機能により認証を行うことで本システムにログインをする.ログアウト機能では本システムにすでにログインしている会員が任意のタイミングでログアウトすることができる.
        \item 退会 \mbox{}\\
        退会機能では会員が任意のタイミングで本システムから退会することのできる機能.退会した会員の情報は退会処理が終了した時点で完全に消去する.
        \item 問い合わせ \mbox{}\\
        一般利用者が質問や要望を運営側に問い合わせることができる機能.質問や要望はメール形式で送信され,運営側からの回答は本システム利用者が登録しているGoogleアカウントのメールアドレスへ送信される.
        \item 事業者登録申請機能 \mbox{}\\
        一般会員から事業者会員への昇格を運営側へ申請する機能.店舗名,電話番号,店舗の住所などの情報を入力し運営からの承認を経て事業者会員となる.
    \end{itemize}
    \item 地図表示・ピン表示機能
    \begin{itemize}[itemsep=10pt]
        \item 地図上でのピン表示 \mbox{}\\
        他の会員が登録したピンを地図上に視覚的に表示する機能.この際,一般会員が投稿したピンは円で事業者会員が投稿したピンは四角で表示される.同一エリア内に存在するピンの数に応じて表示されるピンの大きさが変化する.
        \item ジャンル分け機能 \mbox{}\\
        投稿の内容に応じてジャンルを選択できる機能.ジャンルに応じてピンの色が変化する.
        \item  リアクション機能 \mbox{}\\
        各ピンに対してほかの一般会員からのリアクションを表示する機能.
    \end{itemize}
    \item 絞り込み機能
    \begin{itemize}[itemsep=10pt]
        \item ジャンル \mbox{}\\
        特定のジャンルに属するピンのみを表示する機能.
        \item 日にち \mbox{}\\
        指定した日付または期間内に投稿されたピンのみを表示する機能.
    \end{itemize}
    \item ピン登録機能(匿名)
    \begin{itemize}[itemsep=10pt]
        \item 場所情報登録機能 \mbox{}\\
        一般会員自身が匿名で任意の場所を地図上にピンとして登録できる機能.登録時は位置情報を地図上で指定または企業名で検索により入力.
        \item 記述・写真情報登録機能 \mbox{}\\
        一般会員自身がつけたピンに対して説明文等のテキスト,写真情報を登録できる機能.
        \item 時間情報登録機能 \mbox{}\\
        ピンの登録,記述・写真を投稿した際の時刻を登録する機能.
        \item ジャンル分類登録機能 \mbox{}\\
        投稿内容をジャンルとして分類する機能.
    \end{itemize}
    \item マイページ
    \begin{itemize}[itemsep=10pt]
        \item リアクション履歴閲覧機能 \mbox{}\\
        一般会員自身がこれまでにリアクションを行った投稿の履歴を閲覧できる機能.
        \item 投稿履歴閲覧機能 \mbox{}\\
        一般会員自身がこれまでに登録したピン投稿の履歴を一覧で確認できる機能.
        \item 投稿内容削除機能 \mbox{}\\
        一般会員自身がこれまでに登録したピン投稿を削除できる機能.
    \end{itemize}
\end{itemize}

\subsubsection{事業者向け}
\begin{itemize}[itemsep=10pt]
    \item 会員機能
    \begin{itemize}[itemsep=10pt]
        \item ログイン・ログアウト \mbox{}\\
        ログイン機能では事業者会員がGoogleアカウントを用いた認証機能により認証を行うことで本システムにログインをする.ログアウト機能では本システムにすでにログインしている会員が任意のタイミングでログアウトすることができる.
        \item 退会 \mbox{}\\
        退会機能では会員が任意のタイミングで本システムから退会することのできる機能.退会した会員の情報は退会処理が終了した時点で完全に消去する.
        \item 問い合わせ \mbox{}\\
        事業者会員が質問や要望を運営側に問い合わせることができる機能.質問や要望はメール形式で送信され,運営側からの回答は本システム利用者が登録しているGoogleアカウントのメールアドレスへ送信される.
    \end{itemize}
    \item 地図表示・ピン表示機能
    \begin{itemize}[itemsep=10pt]
        \item 地図上でのピン表示 \mbox{}\\
        他の会員が登録したピンを地図上に視覚的に表示する機能.この際,一般会員が投稿したピンは円で事業者会員が投稿したピンは四角で表示される.同一エリア内に存在するピンの数に応じて表示されるピンの大きさが変化する.
        \item ジャンル分け機能 \mbox{}\\
        投稿の内容に応じてジャンルを選択できる機能.ジャンルに応じてピンの色が変化する.
    \end{itemize}
    \item ピン登録機能(企業名)
    \begin{itemize}[itemsep=10pt]
        \item 場所情報登録機能 \mbox{}\\
        事業者会員自身が匿名で任意の場所を地図上にピンとして登録できる機能.登録時は位置情報を地図上で指定または企業名で検索により入力.
        \item 記述・写真情報登録機能 \mbox{}\\
        事業者会員自身がつけたピンに対して説明文等のテキスト,写真情報を登録できる機能.
        \item 時間情報登録機能 \mbox{}\\
        ピンの登録,記述・写真を投稿した際の時刻を登録する機能.
        \item ジャンル分類登録機能 \mbox{}\\
        投稿内容をジャンルとして分類する機能.
    \end{itemize}
    \item マイページ
    \begin{itemize}[itemsep=10pt]
        \item 投稿履歴閲覧機能 \mbox{}\\
        事業者会員自身がこれまでに登録したピン投稿の履歴を一覧で確認できる機能.
        \item 投稿内容削除機能 \mbox{}\\
        事業者会員自身がこれまでに登録したピン投稿を削除できる機能.
    \end{itemize}
\end{itemize}

\subsubsection{管理者}
\begin{itemize}[itemsep=10pt]
    \item 投稿の削除機能 \mbox{}\\
    管理者が不適切な投稿や規約に違反する投稿を削除できる機能.削除後はシステム上から該当投稿を完全に削除し,該当するアカウントに通知する.
    \item アカウントの削除機能 \mbox{}\\
    管理者が利用規約違反や不正行為を行った全アカウントを凍結できる機能.削除に伴い,関連する投稿・リアクション情報も一括で消去される.
    \item 企業アカウントへの変更 \mbox{}\\
    管理者が申請のあった一般会員アカウントを企業会員アカウントへ変更できる機能.企業認証の確認後,企業向けの機能が利用可能となる.
\end{itemize}

\subsection{前提条件}
本提案書では以下の条件を前提条件とする.
\begin{itemize}
    \item 利用者がインターネットに接続可能な端末 (スマートフォンや PC など) を保有していること.
    \item 利用者が本システムの規約に同意済みであること.
    \item 利用者がGoogleアカウントを所有していること.
    \item 1Googleアカウントに対して本システムの1アカウントの所持のみとすること.
\end{itemize}

\subsection{制約事項}
本システムの制約事項を以下に示す.
\begin{itemize}
    \item 利用者の個人情報漏洩を防ぎ,保護する仕様であること.
    \item 利用者の個人情報を編集できない仕様であること.
    \item 管理者側による投稿の削除・規制が行える仕様であること.
\end{itemize}