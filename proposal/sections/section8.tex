\section{費用対効果}
\subsection{効果}
本システムの導入により,地域住民が身近な出来事やイベント情報を地図上で共有できるようになる.ピン表示機能によって地域内の情報が視覚的に整理され,ジャンルごとの
色分けやリアクション機能を通じて,住民間の関心の傾向や交流の動向が可視化される.これにより,従来は口コミや限定的な SNS投稿に留まっていた地域の小規模な活動や話題
が広く共有され,地域内での情報循環が活発化する.\par
また,ジャンルや日付による絞り込み機能により,利用者は自分の興味関心や予定に合わせて地域イベントや活動を容易に発見できる.清掃活動,ボランティア,祭り,防災訓
練などへの参加が促進され,地域社会における住民参加率と情報発信率の向上が期待できる.さらに,投稿データの蓄積によって,地域ごとの関心分野や活動傾向を定量的に把握
できるため,自治体や地域団体が実証的データに基づく地域施策の立案・改善を行いやすくなる. \par
これらの仕組みにより,地域内の情報流通と住民参加が循環的に強化され,地域社会の協働性および自律的な活動基盤の形成が促進される.

\subsection{費用}
本システムの開発及び運用における費用をそれぞれ表\ref{fig:Q10},\ref{fig:Q11}に示す.
\clearpage

\begin{table}[h]
  \centering
  \caption{開発費}
  \label{fig:Q10}
  \begin{tabular}{crcrc}
  \hline
  項目  & 単価(円) & 数量   & 4ヶ月分換算費用(円) & 備考\\ \hline\hline
  
メインサーバ  & 約5,000円/月 & 2  & 約40,000円&  \\ \hline

データベースサーバ & 約4,000円/月 & 2  & 約32,000円& \\\hline

クラウド(AWS EC2) &約10,000円/月 &1 & 約40,000円& 小規模用を想定\\ \hline

管理者端末(PC) & 約120,000円/台& 1& 約120,000円& 5年間使用想定 \\ \hline

人件費  & 約400,000円/月 & 7& 約11,200,000円& 給料のみを想定\\ \hline\hline

合計 & & & 約11,432,000円\\ \hline
\end{tabular}
\end{table}



\begin{table}[h]
  \centering
  \caption{運用維持費}
  \label{fig:Q11}
  \begin{tabular}{crccr}
  \hline
  項目  & 単価(円) & 数量  & 期間 & 換算費用(円) \\ \hline\hline
 
メインサーバ  & 約5,000円/月 & 2& 60ヶ月  & 約600,000円 \\ \hline

データベースサーバ & 約4,000円/月 &2& 60ヶ月 & 約480,000円 \\\hline

クラウド(AWS EC2)  &約10,000円/月&1 &60ヶ月 & 約600,000円 \\ \hline\hline

合計 &  & & & 約1,680,000円\\ \hline
\end{tabular}
\end{table}

以上より,本システムを開発・5年間運用するために必要な費用は以下である.
\begin{table}[h]
  \centering
  \caption{費用総額}
  \label{fig:Q12}
  \begin{tabular}{ccc}
  \hline
  開発費 & 運用費 & 総額  \\ \hline\hline
 約11,432,000円 & 約1,680,000円 & 約13,112,000円\\ \hline

\end{tabular}
\end{table}


\subsection{収益・利益}
以下の式は,5年間で得られる収益を計算している.5年間でかかる費用は,表\ref{fig:Q12}より,12,572,000円である.\par
土佐山田町全体の事業者数は約1200店舗であり,そのうち広告対事業者は約435店舗である.約435店舗のうち,130店舗が本システムのサブスク機能を利用すると仮定する.

本システムは,企業用を使用する場合は5000円/月の利用料を設定する.この時,130店舗から月々5000円を徴収することで5年間の運用利益は以下になる.\par

\[5000(円)\times 60(ヶ月)\times 130(店舗)=39,000,000円(収益)\]


5年間で見込まれる収益から,5年間でかかる費用を差し引くと,本システムによって得られる5年間の利益は以下になる.

\[39,000,000円(収益)-13,112,000円(開発・運用費)=25,888,000円(利益)\]
これにより2年で開発・運用費を回収することができる.








