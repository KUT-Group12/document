\section{課題解決のための提案}
土佐山田町の地域情報を積極的に入手していないこととイベントや飲食店の情報を入手できるようにするために土佐山田町の情報に特化したSNSを提案する.このSNSにより自治体や,企業,個人の発信をまとめることで情報の集約化を行い,情報収集の効率を上げる.また,このSNSが普及することで地域活性化が期待できる.

\subsection{情報の集約化}
アンケートの結果図\ref{fig:Q1}より,土佐山田町に住んでいる人は地域の情報を積極的に入手している人が少ない.これは,土佐山田町の情報を発信している媒体が複数あり,情報を収集する意欲が起きにくいことが原因であると考えられる.そこで,土佐山田町の情報に特化したSNSを作成することで情報を集約化し,情報収集にかける時間と労力を減らすことが期待できる.

\subsection{情報収集の効率化}
総務省の情報通信白書\cite{label3}のSNSの項目より,日本のソーシャルメディアの利用者は1億580万人となっており,今後も緩やかに増加していくと見られている.使用されているメディアとしてはFacebook,Instagram,X(旧Twitter)が主流である,このことから,主流のソーシャルメディアに1日あたりに投稿される数は膨大であることが予想できる.そのため,情報収集をするには膨大な情報の中から必要な情報を見つけ出さなければならず,効率的でないと言える.そこで,土佐山田町の情報に特化したSNSを作成することで,メディアに存在する情報量を少なくして,情報収集の効率を上げることができると予想できる.

\subsection{地域活性化}
アンケートの結果図\ref{fig:Q4}より,イベント情報や飲食店の情報を知りたいという回答が多かった.また,図\ref{fig:Q7}より,地図上での投稿の閲覧機能を挙げた人が多かった.このことから,地図上にイベント情報や飲食店の情報をできるようにすることで多くの利用者を見込めることになると考える.また,このSNSに投稿されたイベントの情報や飲食店の情報を見た利用者が,実際にイベントへの参加や飲食店への来店をすることで地域活性化が期待できる.